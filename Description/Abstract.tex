% Mengubah keterangan `Abstract` ke bahasa indonesia.
% Hapus bagian ini untuk mengembalikan ke format awal.
\renewcommand\abstractname{Abstract}

\begin{abstract}

  % Ubah paragraf berikut sesuai dengan abstrak dari penelitian.
    % Media Sosial adalah Platform Digital yang memfasilitasi peng-
    % gunannya untuk saling bersosial, baik itu berkomunikasi atau mem-
    % bagikan konten berupa tulisan, foto, dan video. Segala konten yang
    % dibagikan tersebut akan terbuka untuk publik. Media Sosial juga
    % mempunyai banyak dampak, mulai dari dampak positif dan dampak
    % negatif. Salah satu contoh dampak negatif nya adalah ujaran keben-
    % cian dalam berperilaku di media sosial, salah satu wadahnya adalah
    % melalui Twitter. Masyarakat Indonesia dalam bermedia sosial me-
    % miliki perilaku yang kurang baik dan memiliki kemungkinan untuk
    % menyebar kebencian di Twitter. Sehingga, diciptakannya penelitian
    % untuk mendeteksi emosi dari tweet yang ada untuk meminimalisir
    % perilaku kurang baik dalam menggunakan Media Sosial. Peneli-
    % tian ini memanfaatkan BERT sebagai algoritma yang digunakan
    % untuk mendeteksi Emosi Teks apakah sedang marah, senang, se-
    % dih, dan lainnya secara otomatis. Sebelum dilakukan deteksi oleh
    % BERT, teks akan masuk tahap tokenisasi. Keluaran dari klasifikasi
    % menggunakan BERT adalah probabilitas apakah teks tweet memili-
    % ki kecenderungan untuk mempunyai emosi sesuai dengan yang telah
    % diklasifikasikan. Tujuan dari penelitian ini adalah untuk membu-
    % at model yang dapat digunakan untuk melakukan klasifikasi suatu
    % teks untuk memiliki kecenderungan emosi sesuai dengan model yang
    % telah dilatih untuk mengetahui teks memiliki kemungkinan mem-
    % punyai emosi tertentu. Hasil dari penelitian ini adlaah model yang
    % dapat mendeteksi emosi text twitter dengan tingkat akurasi sebesar
    % diatas 80
    
    
    Social Media is a Digital Platform that facilitates the
    use to socialize with each other, be it communicating or
    share content in the form of writing, photos, and videos. All content that
    shared will be open to the public. Social Media too
    has many impacts, ranging from positive and negative impacts
    negative. One example of its negative impact is the utterance of
    cian in behaving on social media, one of the containers is
    via Twitter. Indonesian people in using social media
    have bad behavior and have the possibility to
    spread hate on Twitter. Thus, the creation of research
    to detect emotions from existing tweets to minimize
    bad behavior in using Social Media. researcher-
    This tian utilizes BERT as the algorithm used
    to detect Text Emotions whether they are angry, happy,
    dih, and more automatically. Prior to detection by
    BERT, the text will enter the tokenization stage. Output from classification
    using BERT is the probability of whether the tweet text has
    ki the tendency to have emotions in accordance with what has been
    classified. The aim of this research is to make
    at a model that can be used to classify a
    text to have an emotional tendency according to the model
    have been trained to know the text has the possibility of
    have certain emotions. The results of this study are models that
    can detect twitter text emotions with an accuracy level of
    above 80\%.

\end{abstract}

% Mengubah keterangan `Index terms` ke bahasa indonesia.
% Hapus bagian ini untuk mengembalikan ke format awal.
\renewcommand\IEEEkeywordsname{Keywords}

\begin{IEEEkeywords}

  % Ubah kata-kata berikut sesuai dengan kata kunci dari penelitian.
  BERT, Emotion, Classification, Probability.

\end{IEEEkeywords}
